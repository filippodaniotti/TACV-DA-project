% author: Filippo
\section{Semi-Supervised learning methods}
Some have explored the idea of using SSL methods in UDA: this is achieved by reframing SSL as a special case of UDA. Zhang et al.\cite{Zhang2021} pointed that both UDA and SSL aims to use unlabelled (target) data as data-dependent regularizer to improve model performance over a labelled (source) baseline. 

When it comes to reframing: in a SSL scenario labelled data are typically insufficient to represent the overall distribution \(P_{\textrm{ssl}}\), i.e. labelled data only represent a sub-domain of that distribution. We can consider the sub-domain with the smallest support set for the distribution \(P_{\textrm{small}}\) where all classes are represented; hence, we can consider \(P_{\textrm{small}}\) and \(P_{\textrm{ssl}}\) as source and target distribution in a UDA setting, thereby allowing to run SSL pipelines for UDA tasks. Zhang et al.\cite{Zhang2021} tried several SOTA SSL methods and achieved notable results.